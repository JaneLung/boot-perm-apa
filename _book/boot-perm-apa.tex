\documentclass[]{book}
\usepackage{lmodern}
\usepackage{amssymb,amsmath}
\usepackage{ifxetex,ifluatex}
\usepackage{fixltx2e} % provides \textsubscript
\ifnum 0\ifxetex 1\fi\ifluatex 1\fi=0 % if pdftex
  \usepackage[T1]{fontenc}
  \usepackage[utf8]{inputenc}
\else % if luatex or xelatex
  \ifxetex
    \usepackage{mathspec}
  \else
    \usepackage{fontspec}
  \fi
  \defaultfontfeatures{Ligatures=TeX,Scale=MatchLowercase}
\fi
% use upquote if available, for straight quotes in verbatim environments
\IfFileExists{upquote.sty}{\usepackage{upquote}}{}
% use microtype if available
\IfFileExists{microtype.sty}{%
\usepackage{microtype}
\UseMicrotypeSet[protrusion]{basicmath} % disable protrusion for tt fonts
}{}
\usepackage{hyperref}
\hypersetup{unicode=true,
            pdftitle={APA Statement using Resampling Techniques for GLM using R},
            pdfauthor={Ekarin E. Pongpipat, Phillip F. Agres, Paulina Skolasinska},
            pdfborder={0 0 0},
            breaklinks=true}
\urlstyle{same}  % don't use monospace font for urls
\usepackage{natbib}
\bibliographystyle{apalike}
\usepackage{color}
\usepackage{fancyvrb}
\newcommand{\VerbBar}{|}
\newcommand{\VERB}{\Verb[commandchars=\\\{\}]}
\DefineVerbatimEnvironment{Highlighting}{Verbatim}{commandchars=\\\{\}}
% Add ',fontsize=\small' for more characters per line
\usepackage{framed}
\definecolor{shadecolor}{RGB}{248,248,248}
\newenvironment{Shaded}{\begin{snugshade}}{\end{snugshade}}
\newcommand{\AlertTok}[1]{\textcolor[rgb]{0.94,0.16,0.16}{#1}}
\newcommand{\AnnotationTok}[1]{\textcolor[rgb]{0.56,0.35,0.01}{\textbf{\textit{#1}}}}
\newcommand{\AttributeTok}[1]{\textcolor[rgb]{0.77,0.63,0.00}{#1}}
\newcommand{\BaseNTok}[1]{\textcolor[rgb]{0.00,0.00,0.81}{#1}}
\newcommand{\BuiltInTok}[1]{#1}
\newcommand{\CharTok}[1]{\textcolor[rgb]{0.31,0.60,0.02}{#1}}
\newcommand{\CommentTok}[1]{\textcolor[rgb]{0.56,0.35,0.01}{\textit{#1}}}
\newcommand{\CommentVarTok}[1]{\textcolor[rgb]{0.56,0.35,0.01}{\textbf{\textit{#1}}}}
\newcommand{\ConstantTok}[1]{\textcolor[rgb]{0.00,0.00,0.00}{#1}}
\newcommand{\ControlFlowTok}[1]{\textcolor[rgb]{0.13,0.29,0.53}{\textbf{#1}}}
\newcommand{\DataTypeTok}[1]{\textcolor[rgb]{0.13,0.29,0.53}{#1}}
\newcommand{\DecValTok}[1]{\textcolor[rgb]{0.00,0.00,0.81}{#1}}
\newcommand{\DocumentationTok}[1]{\textcolor[rgb]{0.56,0.35,0.01}{\textbf{\textit{#1}}}}
\newcommand{\ErrorTok}[1]{\textcolor[rgb]{0.64,0.00,0.00}{\textbf{#1}}}
\newcommand{\ExtensionTok}[1]{#1}
\newcommand{\FloatTok}[1]{\textcolor[rgb]{0.00,0.00,0.81}{#1}}
\newcommand{\FunctionTok}[1]{\textcolor[rgb]{0.00,0.00,0.00}{#1}}
\newcommand{\ImportTok}[1]{#1}
\newcommand{\InformationTok}[1]{\textcolor[rgb]{0.56,0.35,0.01}{\textbf{\textit{#1}}}}
\newcommand{\KeywordTok}[1]{\textcolor[rgb]{0.13,0.29,0.53}{\textbf{#1}}}
\newcommand{\NormalTok}[1]{#1}
\newcommand{\OperatorTok}[1]{\textcolor[rgb]{0.81,0.36,0.00}{\textbf{#1}}}
\newcommand{\OtherTok}[1]{\textcolor[rgb]{0.56,0.35,0.01}{#1}}
\newcommand{\PreprocessorTok}[1]{\textcolor[rgb]{0.56,0.35,0.01}{\textit{#1}}}
\newcommand{\RegionMarkerTok}[1]{#1}
\newcommand{\SpecialCharTok}[1]{\textcolor[rgb]{0.00,0.00,0.00}{#1}}
\newcommand{\SpecialStringTok}[1]{\textcolor[rgb]{0.31,0.60,0.02}{#1}}
\newcommand{\StringTok}[1]{\textcolor[rgb]{0.31,0.60,0.02}{#1}}
\newcommand{\VariableTok}[1]{\textcolor[rgb]{0.00,0.00,0.00}{#1}}
\newcommand{\VerbatimStringTok}[1]{\textcolor[rgb]{0.31,0.60,0.02}{#1}}
\newcommand{\WarningTok}[1]{\textcolor[rgb]{0.56,0.35,0.01}{\textbf{\textit{#1}}}}
\usepackage{longtable,booktabs}
\usepackage{graphicx,grffile}
\makeatletter
\def\maxwidth{\ifdim\Gin@nat@width>\linewidth\linewidth\else\Gin@nat@width\fi}
\def\maxheight{\ifdim\Gin@nat@height>\textheight\textheight\else\Gin@nat@height\fi}
\makeatother
% Scale images if necessary, so that they will not overflow the page
% margins by default, and it is still possible to overwrite the defaults
% using explicit options in \includegraphics[width, height, ...]{}
\setkeys{Gin}{width=\maxwidth,height=\maxheight,keepaspectratio}
\IfFileExists{parskip.sty}{%
\usepackage{parskip}
}{% else
\setlength{\parindent}{0pt}
\setlength{\parskip}{6pt plus 2pt minus 1pt}
}
\setlength{\emergencystretch}{3em}  % prevent overfull lines
\providecommand{\tightlist}{%
  \setlength{\itemsep}{0pt}\setlength{\parskip}{0pt}}
\setcounter{secnumdepth}{5}
% Redefines (sub)paragraphs to behave more like sections
\ifx\paragraph\undefined\else
\let\oldparagraph\paragraph
\renewcommand{\paragraph}[1]{\oldparagraph{#1}\mbox{}}
\fi
\ifx\subparagraph\undefined\else
\let\oldsubparagraph\subparagraph
\renewcommand{\subparagraph}[1]{\oldsubparagraph{#1}\mbox{}}
\fi

%%% Use protect on footnotes to avoid problems with footnotes in titles
\let\rmarkdownfootnote\footnote%
\def\footnote{\protect\rmarkdownfootnote}

%%% Change title format to be more compact
\usepackage{titling}

% Create subtitle command for use in maketitle
\providecommand{\subtitle}[1]{
  \posttitle{
    \begin{center}\large#1\end{center}
    }
}

\setlength{\droptitle}{-2em}

  \title{APA Statement using Resampling Techniques for GLM using R}
    \pretitle{\vspace{\droptitle}\centering\huge}
  \posttitle{\par}
  \subtitle{This book is a reference on how to perform resampling techniques (e.g., bootstrapping and permutation testing) to write a more informed APA statement.}
  \author{Ekarin E. Pongpipat, Phillip F. Agres, Paulina Skolasinska}
    \preauthor{\centering\large\emph}
  \postauthor{\par}
      \predate{\centering\large\emph}
  \postdate{\par}
    \date{2019-11-16}

\usepackage{booktabs}

\begin{document}
\maketitle

{
\setcounter{tocdepth}{1}
\tableofcontents
}
\hypertarget{introduction}{%
\chapter{Introduction}\label{introduction}}

Currently, the APA statement includes:

\begin{itemize}
\tightlist
\item
  Estimate
\item
  Name of Statistic
\item
  Degrees of Freedom (df)
\item
  Statistical Value
\item
  \emph{p}-value
\end{itemize}

However, this is not always informative. The APA statement should include:

\begin{itemize}
\tightlist
\item
  Estimate and its Confidence Interval (CI)
\item
  Name of Statistic
\item
  Degrees of Freedom (df)
\item
  Statistical Value
\item
  Mean Squared Error (MSE)
\item
  Adjusted R\^{}2 and its CI
\item
  Permutated P-Value
\end{itemize}

\hypertarget{bootstrap}{%
\chapter{Bootstrap}\label{bootstrap}}

\begin{Shaded}
\begin{Highlighting}[]
\KeywordTok{library}\NormalTok{(tidyverse)}
\NormalTok{df <-}\StringTok{ }\NormalTok{carData}\OperatorTok{::}\NormalTok{Salaries}
\end{Highlighting}
\end{Shaded}

\hypertarget{simple-linear-regression}{%
\section{Simple Linear Regression}\label{simple-linear-regression}}

\begin{Shaded}
\begin{Highlighting}[]
\CommentTok{# set number of bootstraps}
\NormalTok{n_bootstrap <-}\StringTok{ }\DecValTok{1000}

\CommentTok{# create empty dataframes for coefficients and r-squared}
\NormalTok{bootstrap_coef <-}\StringTok{ }\KeywordTok{tibble}\NormalTok{(}\DataTypeTok{n_iter =} \OtherTok{NA}\NormalTok{,}
                         \DataTypeTok{.rows =}\NormalTok{ n_bootstrap)}
\NormalTok{bootstrap_rsq <-}\StringTok{ }\KeywordTok{tibble}\NormalTok{(}\DataTypeTok{n_iter =} \OtherTok{NA}\NormalTok{,}
                         \DataTypeTok{.rows =}\NormalTok{ n_bootstrap)}

\CommentTok{# for loop for bootstrap}
\ControlFlowTok{for}\NormalTok{ (i }\ControlFlowTok{in} \DecValTok{1}\OperatorTok{:}\NormalTok{n_bootstrap) \{}
  
  \CommentTok{# randomly sample with replacement from the rows}
\NormalTok{  idx <-}\StringTok{ }\KeywordTok{sample}\NormalTok{(}\DecValTok{1}\OperatorTok{:}\KeywordTok{nrow}\NormalTok{(df), }\KeywordTok{nrow}\NormalTok{(df), }\DataTypeTok{replace =}\NormalTok{ T)}
\NormalTok{  df_boot <-}\StringTok{ }\NormalTok{df[idx,]}
  
  \CommentTok{# run linear model}
\NormalTok{  model <-}\StringTok{ }\KeywordTok{lm}\NormalTok{(salary }\OperatorTok{~}\StringTok{ }\NormalTok{yrs.since.phd }\OperatorTok{+}\StringTok{ }\NormalTok{yrs.service, df_boot)}
  
  \CommentTok{# extract estimates and r^2 value}
\NormalTok{  summary_model <-}\StringTok{ }\KeywordTok{summary}\NormalTok{(model)}
\NormalTok{  t_stat <-}\StringTok{ }\NormalTok{summary_model}\OperatorTok{$}\NormalTok{coefficients[, }\StringTok{"t value"}\NormalTok{]}
\NormalTok{  df_denom <-}\StringTok{ }\NormalTok{summary_model}\OperatorTok{$}\NormalTok{df[[}\DecValTok{2}\NormalTok{]]}
\NormalTok{  r_sq <-}\StringTok{ }\NormalTok{t_stat}\OperatorTok{^}\DecValTok{2} \OperatorTok{/}\StringTok{ }\NormalTok{(t_stat}\OperatorTok{^}\DecValTok{2} \OperatorTok{+}\StringTok{ }\NormalTok{df_denom)}
  
  \CommentTok{# write bootstrap iteration}
\NormalTok{  bootstrap_coef[i, }\DecValTok{1}\NormalTok{] <-}\StringTok{ }\NormalTok{i}
\NormalTok{  bootstrap_rsq[i, }\DecValTok{1}\NormalTok{] <-}\StringTok{ }\NormalTok{i}
  
  \CommentTok{# determine number of coefficients}
\NormalTok{  n_coef <-}\StringTok{ }\KeywordTok{length}\NormalTok{(model}\OperatorTok{$}\NormalTok{coefficients)}
  
  \CommentTok{# write estimate and r^2 to table for looping across variables}
  \ControlFlowTok{for}\NormalTok{ (j }\ControlFlowTok{in} \DecValTok{1}\OperatorTok{:}\NormalTok{n_coef) \{}
    
    \CommentTok{# bootstrap estimate confidence interval}
\NormalTok{    bootstrap_coef[i, }\KeywordTok{names}\NormalTok{(model}\OperatorTok{$}\NormalTok{coefficients[j])] <-}\StringTok{ }\NormalTok{model}\OperatorTok{$}\NormalTok{coefficients[[j]]}
    
    \CommentTok{# bootstrap R^2 CI}
\NormalTok{    bootstrap_rsq[i, }\KeywordTok{names}\NormalTok{(model}\OperatorTok{$}\NormalTok{coefficients[j])] <-}\StringTok{ }\NormalTok{r_sq[j]}
\NormalTok{  \}}
\NormalTok{\}}

\CommentTok{# print estimate CI and R^2 CI for each variable}
\ControlFlowTok{for}\NormalTok{ (k }\ControlFlowTok{in} \DecValTok{2}\OperatorTok{:}\NormalTok{n_coef}\OperatorTok{+}\DecValTok{1}\NormalTok{) \{}
  
  \CommentTok{# estimate CI}
  \KeywordTok{cat}\NormalTok{(}\StringTok{"}\CharTok{\textbackslash{}n}\StringTok{"}\NormalTok{, }\KeywordTok{colnames}\NormalTok{(bootstrap_coef[,k]), }\StringTok{"}\CharTok{\textbackslash{}n}\StringTok{"}\NormalTok{)}
  \KeywordTok{cat}\NormalTok{(}\StringTok{"}\CharTok{\textbackslash{}n}\StringTok{ estimate CI}\CharTok{\textbackslash{}n}\StringTok{"}\NormalTok{)}
\NormalTok{  bootstrap_ci <-}\StringTok{ }\KeywordTok{quantile}\NormalTok{(}\KeywordTok{as.matrix}\NormalTok{(bootstrap_coef[,k]), }\DataTypeTok{probs =} \KeywordTok{c}\NormalTok{(}\FloatTok{0.025}\NormalTok{, }\FloatTok{.975}\NormalTok{))}
  \KeywordTok{print}\NormalTok{(bootstrap_ci)}
  
  \CommentTok{# R^2 CI}
  \KeywordTok{cat}\NormalTok{(}\StringTok{"}\CharTok{\textbackslash{}n}\StringTok{ R^2 CI}\CharTok{\textbackslash{}n}\StringTok{"}\NormalTok{)}
\NormalTok{  bootstrap_rsq_ci <-}\StringTok{ }\KeywordTok{quantile}\NormalTok{(}\KeywordTok{as.matrix}\NormalTok{(bootstrap_rsq[,k]), }\DataTypeTok{probs =} \KeywordTok{c}\NormalTok{(}\FloatTok{0.025}\NormalTok{, }\FloatTok{.975}\NormalTok{))}
  \KeywordTok{print}\NormalTok{(bootstrap_rsq_ci)}
\NormalTok{\}}
\end{Highlighting}
\end{Shaded}

\begin{verbatim}
## 
##  yrs.since.phd 
## 
##  estimate CI
##      2.5%     97.5% 
##  979.3215 2072.2106 
## 
##  R^2 CI
##       2.5%      97.5% 
## 0.03122567 0.15621742 
## 
##  yrs.service 
## 
##  estimate CI
##         2.5%        97.5% 
## -1162.692563     4.480129 
## 
##  R^2 CI
##         2.5%        97.5% 
## 0.0002034577 0.0535608331
\end{verbatim}

\hypertarget{permutation}{%
\chapter{Permutation}\label{permutation}}

\hypertarget{apa-statement}{%
\chapter{APA Statement}\label{apa-statement}}

We have finished a nice book.

\hypertarget{assumptions}{%
\chapter{Assumptions}\label{assumptions}}

\hypertarget{multiple-comparison-correction}{%
\chapter{Multiple Comparison Correction}\label{multiple-comparison-correction}}

\bibliography{book.bib,packages.bib}


\end{document}
